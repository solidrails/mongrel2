\chapter{Hacking}

This chapter is all about making cool things with Mongrel2.  It covers all the non-deployment
features that you get from the browser side and the handler/backend side of your application.
I'll show you how the chat demo works for the async web sockets.  I'll get into writing your
own handlers using a few other demos.  I'll cover some of the interesting things you can
do with Mongrel2 you can't do with other servers.  Finally I'll get into practical things
when to do proxying and when to use a 0MQ handler.

For the majority of this chapter I'll be using Python, but the demos should translate to
the other languages that are implemented.  I'll periodically show how another language
does one of the demos so you can get the idea that Mongrel2 is \emph{language agnostic}.
In no way should you take me using Python in this chapter to mean you can't use something
else for your handlers.

Currently supported languages are:

\begin{description}
\item [Python] When you installed the \shell{m2sh} gear you also got a \ident{mongrel2} Python library.
\item [Ruby] Ruby support, and probably the most extensive supported language, with good Rack support, by \href{http://github.com/perplexes/m2r}{perplexes on github}.
\item [C++] C++ support by \href{http://github.com/akrennmair/mongrel2-cpp}{akrennmair on github}
\item [PHP] PHP support by \href{http://github.com/winks/m2php}{winks on github}.
\item [C] You can also write handlers in C using the Mongrel2 library, but is really rough, so not recommended yet.  A C library will come though.
\item [Others?] \href{http://zeromq.org}{ZeroMQ} supports Ada, Basic, C, C++, Common Lisp, Erlang, Go, Haskell, Java, Lua, .NET, Objective-C, ooc, Perl, PHP, Python, and Ruby so after reading this chapter you can easily write handlers in any of those languages too.
\end{description}

However, no matter how many languages Mongrel2 supports, you will still have applications that
can't fit into 0MQ handlers and just work better as classic web apps.  Either because you've
already written them and have existing infrastructure, or because of some architectural issues
that require it to run traditionally.  Because of that Mongrel2 supports \emph{HTTP proxying}
which allows you to route requests to basic web server backends that don't support 0MQ.

\begin{aside}{What About FastCGI/AJP/CGI/SCGI/WSGI/Rack?}
Nothing prevents you from writing your own connector between Mongrel2 and your 
deployment protocol of choice.  If you need to run FastCGI or AJP in your environment
then your best bet is to just make a handler that translates Mongrel2 requests
to the protocol you need and back.  The Mongrel2 format is very easy to parse and
translate so you should be able to do it no problem.  The Ruby library already supports
Rack as an example, and Python will support WSGI soon.

However, Mongrel2 itself doesn't support any of these directly.  Doing so would bring
back the language specific infections that cause other web servers to go south.  The 
design of most of these protocols tend to be either before the modern web, or specific
to one particular language.  Instead of trying to cater to all the possible languages
out there, Mongrel2 just gives the tools to connect to it yourself.
\end{aside}



