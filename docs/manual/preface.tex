\chapter{Preface}

This manual will tell you about the most awesome webserver on the planet:
Mongrel2.  It is written for people with a sense of humor who want to get
things done with Mongrel2.  That means, if you're an operations professional,
software developer, hacker or just curious, it's for you.  However, if you're
too serious and think "flowery language" (A.K.A. good entertaining writing) does
not belong in your software manuals, then you should just go read the source
code and save everyone a huge headache dealing with you.

In case you haven't figured it out, this book will be fun and slightly
obnoxious.  That's not intended to insult you, but just to keep you interested
so that you want to read it.


\section{Typography}

Usually the people running the web can be divided into three types of people:  Steves,
Edsgers, and Knuths.

The Steves think that the entire internet should be a wonderful user experience
where all pages are crafted with pixel-perfect fonts with high gloss visuals
and coated with the most happy happy joy joy of all possible experiences.  To
them, design is paramount and actual stability isn't important unless it
interferes with design.  The Steves of the internet think the Edsgers of
internet are destroying the universe with things like "functionality",
"security", and "stability".  Just like the real Steve Jobs, they would rather
everything look fantastic and then use awesome marketing to cover up any
technical flaws.

The Edsgers feel that the internet is completely unsafe, and until it is a fully
curated and crafted set of academic, peer reviewed papers, it will be a festering
pile of dung.  To the Edsgers, the world is dangerous and only a truly paranoid
attitude toward security and stability will ensure that it becomes safe.  They
want every single piece of software to reject all reality and be crafted from
nothing but pure mathematics, and hate the fact that the Steves want to run
around painting the world with useless frivolous colors and words and things
that lead to ambiguity and happiness.

The typography in this book, and the entire project, is for the Knuths of the
world.  I like to think of the Knuths as the practical yet professional types
with a light sense of humor.  They are the ones who are getting things done
while still balancing between great typography and solid bug-free
functionality.  They aren't zealots, but practical, straight-forward type of
people.

That is why this book is written in \TeX, and why it uses whatever fonts \TeX uses.

